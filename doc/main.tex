\documentclass[a4paper]{article}

\usepackage{algorithm}
\usepackage{algorithmic}
\usepackage{booktabs}
\usepackage{amsmath}
\usepackage{amsthm}
\usepackage{amssymb}

\title{Variant-preserving shared prefixes}
\author{Daniel Doerr}

\newtheorem{proposition}{Proposition}
\newtheorem{conjecture}{Conjecture}

\newcommand{\prefix}{\mathbf{p}}
\newcommand{\suffix}{\mathbf{s}}
\newcommand{\orient}{\mathbf{o}}
\newcommand{\parent}{\mathbf{a}}
\newcommand{\child}{\mathbf{c}}
\newcommand{\seq}{\mathbf{t}}
\begin{document}

\maketitle

Let $G = (V, E)$ be a variant graph with node set $V$ representing the set of
\emph{oriented} nodes of $G$. Each node $v \in V$ is associated with a sequence
$\seq(v)$ representing a DNA molecule. For any given sequence $v \in V$ there exists
a unique node $\bar v \in V$ with $\seq(\bar v)$ corresponding to the 
reverse complementary sequence of $\seq(v)$. We further assume that $G$  has a
\emph{source} node $s$ and a \emph{sink} node \emph{S} $s, S \in V$, such that
for any $v \in V$ there exists a path $(s,\ldots, v, \ldots S)$. Then the
\emph{variants} $\mathcal V_G$ of graph $G$ is the set of all possible sequences
$\seq(s).\seq(v_0)\ldots\seq(v_k).\seq(S)$ such that $(s, v_0, \ldots, v_S)$ is
a path in $G$. 

Further, $\prefix(\seq(u), \seq(v))$ denotes the longest common prefix of
sequences $\seq(v)$ and $\seq(u)$. Further, we denote by $\parent(v) := \{u
\mid \{\bar v, u\} \in E\}$ the \emph{parents} of $v$ and by $\child(v) := \{u
\mid \{v, u\} \in E\}$ the \emph{children} of $v$. A triple $(u, v, w)$ with $u
\neq w$ and $\{v, w\} \subseteq\child(u)$ is a \emph{cherry}. 

\begin{proposition}
    Let $(u, v, w)$ be a cherry with $|\prefix(\seq(u), \seq(v))| > 0$. Let $G' =
    (V', E')$ be a graph with 
    \[
        V' = V \setminus \{v, \bar v, w, \bar w\} \cup \{x, \bar x, v', \bar
        v', w', \bar w'\}
    \]
    and 
    \begin{align*}
        E' = & E \setminus (\{\{v, y\} \mid y \in V\} \cup \{w, z\} \mid z \in
        V\}) \\
        & \cup \{\{x, p\} \mid p \in \parent(v) \cup \parent(w)\} \cup
        \{\{\bar x, \bar p \} \mid \bar p \in \parent(\bar v) \cup \parent(\bar
        w) \} \\
        & \cup \{\{x, v'\}, \{x, w'\}, \{\bar x, \bar v'\}, \{\bar x, \bar
        w'\}\} \\
        & \cup \{\{v', c\} \mid c \in \child(v) \} \cup \{\{w', c\} \mid c \in
        \child(w)\}\,,
    \end{align*}
    where $\seq(x) = \prefix(\seq(u), \seq(v))$, and $\seq(v'), \seq(w')$ the
    sequences of $v$ and $w$ without $\prefix(\seq(u), \seq(v))$. 
    Then $\mathcal V_G = \mathcal V_{G'}$ holds true in general if and only if
    $\parent(v) = \parent(w)$.
\end{proposition}

\begin{proof}
    \begin{enumerate}
        \item[$\Rightarrow$] Let $\mathcal V_v := 
            \{\seq(s)\ldots\seq(p).\seq(v).\seq(c)\ldots\seq(S) \mid p \in
            \parent(v), c \in \child(v) \} \subseteq \mathcal V_{G}$ and 
            $V_{v'} :=
            \{\seq(s)\ldots\seq(p).\seq(x).\seq(v').\seq(c)\ldots\seq(S)) \mid
            p \in \parent(x), c \in \child(v') \}\subseteq \mathcal V_{G'}$.
            By construction, we have $\parent(x) = \parent(v)$ and $\child(v')
            = \child(v)$ and $\seq(v) = \seq(x).\seq(v')$, therefore $V_{v'} =
            \{\seq(s)\ldots\seq(p).\seq(v).\seq(c)\ldots\seq(S)) \mid p \in
            \parent(v), c \in \child(v) \} = \mathcal V_v$. The same logic,
            applied to all other new variants of $\mathcal V_{G'}$, completes
            the proof. 

    \item[$\Leftarrow$] Assume $\mathcal V_G = \mathcal V_{G'}$ and $\parent(v)
        \subset \parent(w)$. Consider some path of graph $G$ of the form $(s,
        \ldots, p)$, $p \in \parent(w) \setminus \parent(v)$, s.t.
        $\seq(s)\ldots\seq(p) \not\in \{\seq(s)\ldots\seq(p') \mid p' \in
        \parent(v)\}$. Then the set of sequences
        $\{\seq(s)\ldots\seq(p).\seq(x).\seq(v').\seq(c)\ldots\seq(S) \mid c
        \in \child(v)\}$ is a subset of $\mathcal V_{G'}$, but not of $\mathcal
        V_{G}$, thus contradicting the assumption. 
    \end{enumerate}
\end{proof}

%\begin{conjecture} 
%    Let $G$ be a variant graph s.t. there exists no cherry $(u, v, w)$ be a
%    cherry with $|\prefix(u, v)| > 0$ with $\parent(u) = \parent(v)$. Then
%    there exists no two paths $p = (s, \ldots, S)$ and $p' = (s, \ldots, S)$
%    with $p \neq p'$ and 
%\end{conecture}
\end{document}
